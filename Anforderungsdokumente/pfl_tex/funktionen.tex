\chapter{Produktfunktionen}

\newcounter{pfc}\setcounter{pfc}{10}

\begin{description}[leftmargin=3em, style=sameline]
	
	\begin{php}{pfc}{F}{proz:regist}{pproz:regist} 
		\item [Geschäftsprozess:] Spieler registrieren
		\item [Vorbedingung:] Netzwerkverbindung zum Server
		\item [Nachbedingung Erfolg:] Spieler registriert
		\item [Nachbedingung Fehlschlag:] Spieler nicht registriert
		\item [Akteure:] Spieler
		\item [Auslösendes Ereignis:] Ein neuer Spieler möchte sich registrieren.
		\item [Beschreibung:]\hfill\\
		1 Verfügbarkeit des Namens prüfen \\
		2 Passwort auf Korrektheit prüfen (durch wiederholte Eingabe) \\
		3 Neuen Spieler anlegen
		\item [Erweiterungen:] -
		\item [Alternativen:]\hfill\\
		2a Neuen Spieler ablehnen; Benutzername schon vergeben \\
		3a Neuen Spieler ablehnen; Passwörter stimmen nicht überein
		
		
	\end{php}
	
	\begin{php}{pfc}{F}{proz:entfernen}{pproz:entfernen}
		\item [Geschäftsprozess:] Spieler entfernen
		\item [Vorbedingung:]\hfill\\
		1 Spieler bereits registriert \\
		2 Spieler ist angemeldet
		\item [Nachbedingung Erfolg:] Spieler entfernt
		\item [Nachbedingung Fehlschlag:] Spieler nicht entfernt
		\item [Akteure:] Spieler
		\item [Auslösendes Ereignis:] Ein Spieler möchte seinen Account löschen.
		\item [Beschreibung:] \hfill\\
		1 Passwort abfragen \\
		2 Logout des Spielers und zum Anmeldebildschirm zurückkehren \\
		3 Spielerdaten löschen
		\item [Erweiterungen:] -
		\item [Alternativen:]\hfill\\
		2a Löschvorgang abbrechen, da Passwort nicht gültig
	\end{php}
	
	
	\begin{php}{pfc}{F}{proz:anmeld}{pproz:anmeld}
		\item [Geschäftsprozess:] Spieler anmelden
		\item [Vorbedingung:]\hfill\\
		1 Netzwerkverbindung zum Server \\
		2 Spieler registriert
		\item [Nachbedingung Erfolg:] Spieler angemeldet
		\item [Nachbedingung Fehlschlag:] Spieler nicht angemeldet
		\item [Akteure:] Spieler
		\item [Auslösendes Ereignis:] Ein Spieler möchte sich anmelden.
		\item [Beschreibung:]\hfill\\
		1 Spielername und Passwort überprüfen \\
		2 Lobbyfenster wird geöffnet 
		\item [Erweiterungen:] -
		\item [Alternativen:]\hfill\\
		2a Anmeldung ablehnen; fehlerhafte Anmeldedaten 
	\end{php}
	
	\begin{php}{pfc}{F}{proz:abmeld}{pproz:abmeld}
		\item [Geschäftsprozess:] Spieler abmelden	
		\item [Vorbedingung:]\hfill\\
		1 Spieler ist angemeldet
		2 Spieler ist in der Lobby
		\item [Nachbedingung Erfolg:] Spieler abgemeldet
		\item [Nachbedingung Fehlschlag:] Spieler nicht abgemeldet
		\item [Akteure:] Spieler
		\item [Auslösendes Ereignis:] Ein Spieler möchte sich abmelden.
		\item [Beschreibung:] \hfill\\
		1 Anmeldebildschirm öffnen
		\item [Erweiterungen:] -
		\item [Alternativen:]-		
	\end{php}
	
	\begin{php}{pfc}{F}{proz:erstel}{pproz:erstel}
		\item [Geschäftsprozess:] Spielraum erstellen
		\item [Vorbedingung:]\hfill\\
		1 Spieler ist angemeldet \\
		2 Spieler ist in der Lobby
		\item [Nachbedingung Erfolg:] Spielraum erstellt
		\item [Nachbedingung Fehlschlag:] Spielraum nicht erstellt
		\item [Akteure:] Spieler
		\item [Auslösendes Ereignis:] Ein Spieler möchte einen neuen Spielraum erstellen.
		\item [Beschreibung:]\hfill\\
		1 Verfügbarkeit des Spielraumnamen prüfen \\
		2 Neuen Spielraum anlegen \\
		3 Spielraum in der Spielraumliste anzeigen
		\item [Erweiterungen:] -
		\item [Alternativen:] 2a Neuen Spielraum ablehnen; Spielraumname bereits vergeben
	\end{php}
	
	\begin{php}{pfc}{F}{proz:schliessen}{pproz:schliessen}
		\item [Geschäftsprozess:] Spielraum schließen
		\item [Vorbedingung:]\hfill\\
		1 Spielraum vorhanden \\
		2 Spiel muss abgeschlossen sein
		\item [Nachbedingung Erfolg:] Spielraum geschlossen
		\item [Nachbedingung Fehlschlag:] Spielraum nicht geschlossen
		\item [Akteure:] -
		\item [Auslösendes Ereignis:] Kein menschlicher Spieler mehr im Spielraum. 
		\item [Beschreibung:] \hfill\\
		1 Überprüfe Anzahl menschlicher Spieler \\
		2 Spielraum aus Spielraumliste löschen
		\item [Erweiterungen:] -
		\item [Alternativen:] 2a Spielraum nicht schließen; menschlicher Spieler vorhanden 
	\end{php}
	
	\begin{php}{pfc}{F}{proz:uebert}{pproz:uebert}
		\item [Geschäftsprozess:] Spielraumliste übertragen
		\item [Vorbedingung:]\hfill\\
		Spieler befindet sich in der Lobby
		\item [Nachbedingung Erfolg:] Spielraumliste wurde übertragen
		\item [Nachbedingung Fehlschlag:] -
		\item [Akteure:] -
		\item [Auslösendes Ereignis:] -
		\item [Beschreibung:] \hfill\\
		Spielraumdaten werden übertragen und angezeigt.
		\item [Erweiterungen:] -
		\item [Alternativen:] -
	\end{php}
	
	\begin{php}{pfc}{F}{proz:betret}{pproz:betret}
		\item [Geschäftsprozess:] Spielraum beitreten
		\item [Vorbedingung:]\hfill\\
		1 Spieler ist in der Lobby \\
		2 Spieler befindet sich in keinem anderen Spielraum
		\item [Nachbedingung Erfolg:] Spieler ist dem Spielraum beigetreten
		\item [Nachbedingung Fehlschlag:] Spieler ist dem Spielraum nicht beigetreten
		\item [Akteure:] Spieler
		\item [Auslösendes Ereignis:] Ein Spieler möchte einem Spielraum beitreten.
		\item [Beschreibung:]\hfill\\
		1 Überprüfen, ob Spielraum bereits voll \\
		2 Spielraumfenster öffnet sich
		\item [Erweiterungen:] -
		\item [Alternativen:] 2a Beitritt ablehnen; Raum bereits voll
	\end{php}
	
	\begin{php}{pfc}{F}{proz:verlas}{pproz:verlas}
		\item [Geschäftsprozess:] Spielraum verlassen
		\item [Vorbedingung:]\hfill\\
		1 Spieler ist angemeldet \\
		2 Spieler befindet sich in Spielraum \\
		3 Spiel abgeschlossen
		\item [Nachbedingung Erfolg:] Spieler hat Spielraum verlassen.
		\item [Nachbedingung Fehlschlag:] -
		\item [Akteure:] Spieler
		\item [Auslösendes Ereignis:] Ein Spieler möchte einen Spielraum nach Beenden eines Spiels verlassen.
		\item [Beschreibung:] \hfill\\
		1 Lobby wird geöffnet \\
		2 Spieler wird aus dem Spiel entfernt.
		\item [Erweiterungen:] - 
		\item [Alternativen:] -
	\end{php}
	
	\begin{php}{pfc}{F}{proz:anzeig}{pproz:anzeig}
		\item [Geschäftsprozess:] Gewinnpunkte anzeigen
		\item [Vorbedingung:]\hfill\\
		Spieler im Spiel 
		\item [Nachbedingung Erfolg:] Gewinnpunkte anzeigen
		\item [Nachbedingung Fehlschlag:] Gewinnpunkte werden nicht angezeigt
		\item [Akteure:] -
		\item [Auslösendes Ereignis:] Spielpartie beendet
		\item [Beschreibung:]\hfill\\
		Gewinnpunkte aller Spieler anzeigen
		\item [Erweiterungen:] -
		\item [Alternativen:] -
	\end{php}
	
	\begin{php}{pfc}{F}{proz:anzeig2}{pproz:anzeig2}
		\item [Geschäftsprozess:] Highscore anzeigen
		\item [Vorbedingung:]\hfill\\
		Spieler in der Lobby
		\item [Nachbedingung Erfolg:] Highscore wird angezeigt
		\item [Nachbedingung Fehlschlag:] Highscore wird nicht angezeigt
		\item [Akteure:] Spieler
		\item [Auslösendes Ereignis:] Ein Spieler möchte sich den Highscore aller Spieler in der Lobby anzeigen lassen.
		\item [Beschreibung:]\hfill\\
		1 Highscorefenster öffnen  \\
		2 Highscoreliste anzeigen
		\item [Erweiterungen:] -
		\item [Alternativen:] - 
	\end{php}
	
	\begin{php}{pfc}{F}{proz:statistikberechnen}{pproz:statistikberechnen}
		\item [Geschäftsprozess:] Highscore berechnen
		\item [Vorbedingung:]\hfill\\
		Spieler ist in der Lobby 
		\item [Nachbedingung Erfolg:] Highscore wird berechnet
		\item [Nachbedingung Fehlschlag:] Highscore wird nicht berechnet
		\item [Akteure:] -
		\item [Auslösendes Ereignis:] Spiel beendet
		\item [Beschreibung:] \hfill\\
		Highscore aus den Gewinnpunkten und der Anzahl der Spiele ermitteln.
		\item [Erweiterungen:] -
		\item [Alternativen:] -
	\end{php}
	
	\begin{php}{pfc}{F}{proz:anzeig3}{pproz:anzeig3}
		\item [Geschäftsprozess:] Punkte anzeigen
		\item [Vorbedingung:]\hfill\\
		Spieler befindet sich im Spiel
		\item [Nachbedingung Erfolg:] Punktestand wird angezeigt
		\item [Nachbedingung Fehlschlag:] Punktestand wird nicht angezeigt
		\item [Akteure:] -
		\item [Auslösendes Ereignis:] Spielpartie beendet
		\item [Beschreibung:]\hfill\\
		Punkte des Spielers anzeigen
		\item [Erweiterungen:] -
		\item [Alternativen:] -
	\end{php}
	
	\begin{php}{pfc}{F}{proz:punkteberechnen}{pproz:punkteberechnen}
		\item [Geschäftsprozess:] Punkte berechnen
		\item [Vorbedingung:] \hfill\\
		Spieler im Spiel
		\item [Nachbedingung Erfolg:] Punkte berechnet
		\item [Nachbedingung Fehlschlag:] Punkte nicht berechnet
		\item [Akteure:] -
		\item [Auslösendes Ereignis:] Spielpartie beendet
		\item [Beschreibung:] \hfill\\
		Punkte werden gezählt
		\item [Erweiterungen:] -
		\item [Alternativen:] -
	\end{php}
	
	\begin{php}{pfc}{F}{proz:gewinnpunkteberechnen}{pproz:gewinnpunkteberechnen}
		\item [Geschäftsprozess:] Gewinnpunkte berechnen
		\item [Vorbedingung:]\hfill\\
		Spieler im Spiel
		\item [Nachbedingung Erfolg:] Gewinnpunkte werden berechnet
		\item [Akteure:] -
		\item [Nachbedingung Fehlschlag:] Gewinnpunkte werden nicht berechnet
		\item [Auslösendes Ereignis:] Spielpartie beendet
		\item [Beschreibung:] \hfill\\
		1 Funktion \ref{pproz:punkteberechnen} ausführen \\
		2 Gewinnpunkte ermitteln
		\item [Erweiterungen:] -
		\item [Alternativen:] -
	\end{php}
	
	\begin{php}{pfc}{F}{proz:hinzuf}{pproz:hinzuf}
		\item [Geschäftsprozess:] Bot hinzufügen
		\item [Vorbedingung:]\hfill\\
		1 Admin befindet sich im Spielraum \\
		2 Spiel noch nicht gestartet
		\item [Nachbedingung Erfolg:] Bot hinzugefügt
		\item [Nachbedingung Fehlschlag:] Bot nicht hinzugefügt
		\item [Akteure:] Admin
		\item [Auslösendes Ereignis:] Der Admin möchte einen Bot im Spielraum hinzufügen.
		\item [Beschreibung:]\hfill\\
		1 Überprüfe ob Spielraum voll \\
		2 Neuen Bot im Spielraum erstellen \\
		3 Spieleranzahl erhöhen
		\item [Erweiterungen:] 2a Vorgang abbrechen, falls Spielraum voll
		\item [Alternativen:] 
	\end{php}
	
	\begin{php}{pfc}{F}{proz:entfer}{pproz:entfer}
		\item [Geschäftsprozess:] Bot entfernen
		\item [Vorbedingung:]\hfill\\
		1 Admin befindet sich im Spielraum \\
		2 Spiel noch nicht gestartet \\
		\item [Nachbedingung Erfolg:] Bot entfernt
		\item [Nachbedingung Fehlschlag:] Bot nicht entfernt
		\item [Akteure:] Admin
		\item [Auslösendes Ereignis:] Der Admin möchte einen Bot im Spielraum entfernen.
		\item [Beschreibung:]\hfill\\
		1 Überprüfe ob Bot im Spielraum \\
		2 Bot aus Spielraum entfernen \\
		3 Spieleranzahl verringern
		\item [Erweiterungen:] -
		\item [Alternativen:] -
	\end{php}
	
	\begin{php}{pfc}{F}{proz:notfallbot}{pproz:notfallbot}
		\item [Geschäftsprozess:] Notfallbot springt ein
		\item [Vorbedingung:]\hfill\\
		Spieler befindet sich im laufenden Spiel
		\item [Nachbedingung Erfolg:] Notfallbot eingesprungen
		\item [Nachbedingung Fehlschlag:] Notfallbot nicht eingesprungen
		\item [Akteure:] -
		\item [Auslösendes Ereignis:]  Timer abgelaufen
		\item [Beschreibung:]\hfill\\
		1 Reaktion des Spielers mittels Timer prüfen \\
		2 Notfallbot erstellen \\
		3 Spieler durch Notfallbot ersetzen
		\item [Erweiterungen:] -
		\item [Alternativen:] 2a Vorgang abbrechen, falls Spieler reagiert
	\end{php}
	
	\begin{php}{pfc}{F}{proz:chatte}{pproz:chatte1}
		\item [Geschäftsprozess:] Chatnachricht im Spielraum senden
		\item [Vorbedingung:]\hfill\\
		1 Spieler im Spielraum \\
		2 Nachricht nicht leer
		\item [Nachbedingung Erfolg:] Nachricht gesendet
		\item [Nachbedingung Fehlschlag:] Nachricht nicht gesendet
		\item [Akteure:] Spieler
		\item [Auslösendes Ereignis:] Ein Spieler möchte seinen Mitspielern eine Nachricht schicken.
		\item [Beschreibung:] \hfill\\
		1 Nachricht aus dem Eingabefeld entfernen \\
		2 Nachricht im Chatfenster anzeigen
		\item [Erweiterungen:] -
		\item [Alternativen:] -
	\end{php}
	
	%	\begin{php}{pfc}{F}{proz:chatte}{pproz:chatte1b}
	%		\item [Geschäftsprozess:] Chatnachricht im Spiel empfangen
	%		\item [Vorbedingung:]\hfill\\
	%			1 Netzwerkverbindung zum Server \\
	%			2 Spieler befindet sich im Spiel
	%		\item [Nachbedingung Erfolg:] Nachricht empfangen
	%		\item [Nachbedingung Fehlschlag:] Nachricht nicht empfangen
	%		\item [Akteure:] -
	%		\item [Auslösendes Ereignis:] Ein Spieler möchte im Spiel mit seinen %Mitspielern chatten.
	%		\item [Beschreibung:] Nachricht empfangen
	%		\item [Erweiterungen:] -
	%		\item [Alternativen:] -	
	%	\end{php}
	
	\begin{php}{pfc}{F}{proz:chatte}{pproz:chatte2}
		\item [Geschäftsprozess:] Chatnachricht in der Lobby senden
		\item [Vorbedingung:]\hfill\\
		1 Spieler befindet sich in der Lobby \\
		2 Nachricht ist nicht leer
		\item [Nachbedingung Erfolg:] Nachricht gesendet
		\item [Nachbedingung Fehlschlag:] Nachricht nicht gesendet
		\item [Akteure:] Spieler
		\item [Auslösendes Ereignis:] Ein Spieler möchte in der Lobby eine Nachricht versenden.
		\item [Beschreibung:] \hfill\\
		1 Nachricht aus dem Eingabefeld entfernen \\
		2 Nachricht im Chatfenster anzeigen
		\item [Erweiterungen:] -
		\item [Alternativen:] -	
	\end{php}
	
	%	\begin{php}{pfc}{F}{proz:chatte}{pproz:chatte2b}
	%		\item [Geschäftsprozess:] Chatnachricht in der Lobby empfangen
	%		\item [Vorbedingung:]\hfill\\
	%			1 Netzwerkverbindung zum Server \\
	%			2 Spieler befindet sich in der Lobby
	%		\item [Nachbedingung Erfolg:] Nachricht empfangen
	%		\item [Nachbedingung Fehlschlag:] Nachricht nicht empfangen
	%		\item [Akteure:] -
	%		\item [Auslösendes Ereignis:] Ein Spieler möchte im Spiel mit seinen %Mitspielern Chatten.
	%		\item [Beschreibung:] Nachricht empfangen
	%		\item [Erweiterungen:] -
	%		\item [Alternativen:] -	
	%	\end{php}
	
	
	\begin{php}{pfc}{F}{proz:starte}{pproz:starte}
		\item [Geschäftsprozess:] Spiel starten
		\item [Vorbedingung:]\hfill\\
		1 Spielraum bereits erstellt \\
		2 Spielraum  voll
		\item [Nachbedingung Erfolg:] Spiel gestartet
		\item [Nachbedingung Fehlschlag:] Spiel nicht gestartet
		\item [Akteure:] Admin
		\item [Auslösendes Ereignis:] Der Admin möchte ein Spiel innerhalb eines Spielraums starten
		\item [Beschreibung:]\hfill\\
		1 Spielfenster öffnet sich \\
		2 Gewinnpunkte werden mit 0 initialisiert \\
		3 Die Funktion \ref{pproz:start} ausführen 
		\item [Erweiterungen:] -
		\item [Alternativen:] -
	\end{php}
	
	\begin{php}{pfc}{F}{proz:start}{pproz:start}
		\item [Geschäftsprozess:] Neue Spielpartie starten
		\item [Vorbedingung:]\hfill\\
		Spiel gestartet 
		\item [Nachbedingung Erfolg:] Spielpartie gestartet
		\item [Nachbedingung Fehlschlag:] Spielpartie nicht gestartet
		\item [Akteure:] -
		\item [Auslösendes Ereignis:] Timer abgelaufen
		\item [Beschreibung:] \hfill\\
		1 Die Funktionen \ref{pproz:mischen}, \ref{pproz:austei} und \ref{pproz:trumpf} ausführen \\
		2 Punkte mit 0 initialisieren
		\item [Erweiterungen:] -
		\item [Alternativen:] -
	\end{php}
	
	\begin{php}{pfc}{F}{proz:mischen}{pproz:mischen}
		\item [Geschäftsprozess:] Karten mischen
		\item [Vorbedingung:] -
		\item [Nachbedingung Erfolg:] Karten gemischt
		\item [Nachbedingung Fehlschlag:] -
		\item [Akteure:] -
		\item [Auslösendes Ereignis:] Starten einer neuen Spielpartie innerhalb eines Spiels.
		\item [Beschreibung:] Spielkarten mischen
		\item [Erweiterungen:] -
		\item [Alternativen:] -
	\end{php}
	
	\begin{php}{pfc}{F}{proz:sortieren}{pproz:sortieren}
		\item [Geschäftsprozess:] Handkarten sortieren
		\item [Vorbedingung:]\hfill\\
		Karten wurden ausgeteilt
		\item [Nachbedingung Erfolg:] Handkarten sortiert
		\item [Nachbedingung Fehlschlag:] -
		\item [Akteure:] Spieler
		\item [Auslösendes Ereignis:] Der Spieler möchte seine Karten sortieren lassen.
		\item [Beschreibung:] Karten von links nach rechts nach Wert sortieren. 
		\item [Erweiterungen:] -
		\item [Alternativen:] -
	\end{php}
	
	\begin{php}{pfc}{F}{proz:austei}{pproz:austei}
		\item [Geschäftsprozess:] Karten austeilen
		\item [Vorbedingung:]\hfill\\
		Spielpartie gestartet
		\item [Nachbedingung Erfolg:] Karten ausgeteilt
		\item [Nachbedingung Fehlschlag:] -
		\item [Akteure:] -
		\item [Auslösendes Ereignis:] Spielpartie gestartet
		\item [Beschreibung:]\hfill\\
		Karten werden vom Stapel genommen und den Spieler zugeordnet
		\item [Erweiterungen:] -
		\item [Alternativen:] -	
	\end{php}
	
	\begin{php}{pfc}{F}{proz:ausspi}{pproz:ausspi}
		\item [Geschäftsprozess:] Karte ablegen
		\item [Vorbedingung:] \hfill\\
		1 Spieler im Spiel \\
		2 Karten ausgeteilt \\
		3 Spiel ist am Zug
		\item [Nachbedingung Erfolg:] Karte abgelegt
		\item [Nachbedingung Fehlschlag:] Karte nicht abgelegt
		\item [Akteure:] Spieler
		\item [Auslösendes Ereignis:] Ein Spieler möchte eine gewählte Handkarte ablegen.
		\item [Beschreibung:]\hfill\\
		1 Prüfen, ob gewählte Karte zulässig \\
		2 Karte aus Hand entfernen \\
		3 Karte dem Stich hinzufügen
		\item [Erweiterungen:] -
		\item [Alternativen:] 2a Karte unzulässig; Karte nicht ablegen	
	\end{php}
	
	
	
	\begin{php}{pfc}{F}{proz:trumpf}{pproz:trumpf}
		\item [Geschäftsprozess:] Trumpfkarte zeigen
		\item [Vorbedingung:] \hfill\\
		1 Spieler im Spiel \\
		2 Karten ausgeteilt
		\item [Nachbedingung Erfolg:] Trumpfkarte anzeigen
		\item [Nachbedingung Fehlschlag:] Trumpfkarte nicht anzeigen
		\item [Akteure:] -
		\item [Auslösendes Ereignis:] Alle Karten wurden ausgeteilt.
		\item [Beschreibung:] Trumpfkarte wird allen Spielern während des Spiels angezeigt
		\item [Erweiterungen:] -
		\item [Alternativen:] -
	\end{php}	
	
	\begin{php}{pfc}{F}{proz:drueck}{pproz:drueck}
		\item [Geschäftsprozess:] Karte drücken 
		\item [Vorbedingung:]\hfill\\
		1 Karten ausgeteilt\\
		2 Spieler im Spiel
		\item [Nachbedingung Erfolg:] Karte gedrückt
		\item [Nachbedingung Fehlschlag:] -
		\item [Akteure:] Spieler
		\item [Auslösendes Ereignis:] Der Spieler möchte eine Karte drücken
		\item [Beschreibung:]\hfill\\
		1 Karte verdeckt hinlegen \\
		2 Karte aus Hand entfernen
		\item [Erweiterungen:] -
		\item [Alternativen:] -
	\end{php}
	
	
	\begin{php}{pfc}{F}{proz:stichb}{pproz:stichb}
		\item [Geschäftsprozess:] Stich beenden
		\item [Vorbedingung:]
		Spieler in einer Spielpartie
		\item [Nachbedingung Erfolg:] Stich beendet
		\item [Nachbedingung Fehlschlag:] Stich nicht beendet
		\item [Akteure:] -
		\item [Auslösendes Ereignis:] Alle Spieler haben ihre Karte gelegt
		\item [Beschreibung:] Die Karten werden dem Gewinner zugeteilt
		\item [Erweiterungen:] -
		\item [Alternativen:] -
	\end{php}
	
	\begin{php}{pfc}{F}{proz:partieb}{pproz:partieb}
		\item [Geschäftsprozess:] Spielpartie beenden
		\item [Vorbedingung:]\hfill\\
		Spieler befindet sich im Spiel
		\item [Nachbedingung Erfolg:] Spielpartie beendet
		\item [Nachbedingung Fehlschlag:] -
		\item [Akteure:] -
		\item [Auslösendes Ereignis:] Alle 11 Stiche wurden beendet
		\item [Beschreibung:] \hfill\\
		1 Die Funktionen \ref{pproz:gewinnpunkteberechnen}, \ref{pproz:punkteberechnen},\ref{pproz:anzeig3} und \ref{pproz:anzeig} ausführen
		2 Timer starten
		\item [Erweiterungen:] -
		\item [Alternativen:] -
	\end{php}
	
	\begin{php}{pfc}{F}{proz:spielb}{pproz:spielb}
		\item [Geschäftsprozess:] Spiel beenden
		\item [Vorbedingung:]
		Spieler im Spiel
		\item [Nachbedingung Erfolg:] Spiel beendet
		\item [Nachbedingung Fehlschlag:] -
		\item [Akteure:] -
		\item [Auslösendes Ereignis:] Ein Spieler hat nach den  Eidexregeln das Turnier gewonnen.
		\item [Beschreibung:]\hfill\\
		1 Die Funktion \ref{pproz:statistikberechnen} ausführen \\
		2 Spieler gelangen in den Spielraum
		\item [Erweiterungen:] -
		\item [Alternativen:] -
	\end{php}
	
	\begin{php}{pfc}{F}{proz:regelnanzeigen}{pproz:regelnanzeigen}
		\item [Geschäftsprozess:] Spielregeln anzeigen
		\item [Vorbedingung:] Spieler ist angemeldet-
		\item [Nachbedingung Erfolg:] Spielregeln anzeigen
		\item [Nachbedingung Fehlschlag:] Spielregeln nicht anzeigen
		\item [Akteure:] Spieler
		\item [Auslösendes Ereignis:] Ein Spieler möchte sich die Spielregeln anzeigen lassen.
		\item [Beschreibung:] Ein neues Fenster mit den Spielregeln öffnet sich.
		\item [Erweiterungen:] -
		\item [Alternativen:] -
	\end{php}
	
	
\end{description}