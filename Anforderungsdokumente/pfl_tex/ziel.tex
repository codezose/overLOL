\chapter{Zielbestimmung}

	Im Rahmen des Software-Entwicklungs-Projekts/Modellierungspraktikums 2015 soll ein einfach zu bedienendes Client-Server-System zum Kartenspielen über ein Netzwerk implementiert werden. Die Benutzeroberfläche soll intuitiv bedienbar sein.


	\section{Musskriterien}

		\begin{itemize}
			\item Spieler registrieren / entfernen
			\item Am Server an- und abmelden
			\item Räume verwalten (erstellen / schließen) und ihnen beitreten / sie verlassen
			\item Spielstand, Highscore und Punktestand anzeigen
			\item Bots verwalten (hinzufügen, entfernen), Notfallbot
			\item Spielübergreifende Chatfunktion und gesonderter Chat pro Raum
			\item Spiel nach Eidex Regeln spielen
			\item Handkarten sortieren
			\item Trumpfkarte anzeigen
			\item Computergegner können statt menschlicher Spieler eingesetzt werden
			\item Plattformunabhängigkeit
			\item Quellcode ausführlich dokumentieren und testen (JavaDoc und JUnit)
			\item Spielregeln anzeigen (in der Lobby und im Spiel)
		\end{itemize}


	\section{Wunschkriterien}

		\begin{itemize}
			\item Freundesliste verwalten (Freunde hinzufügen / entfernen)
			\item Freund zum Spiel einladen
			\item Spielerdaten bearbeiten
			\item Soundeffekte
			\item Bots in verschiedenen Schwierigkeitsstufen
			\item in Facebook teilen
			\item Spielraum passwortgeschützt
		\end{itemize}

 	\section{Abgrenzungskriterien} % Wenn nötig

 		\begin{itemize}
 			\item keine nextGen Grafik
 			\item keine kommerzielle Verwendung
 			\item nur deutsche Sprache

		\end{itemize}
