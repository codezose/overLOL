\begin{itemize}
		\item Interoperabilität: Die Software ist in sich geschlossen und benötigt keine Kommunikation mit anderen Softwaresystemen.
		\item Safety: Offensichtlich können durch das Spiel keine Menschen (körperlich) verletzt werden.
		\item Security: Da das Spiel über ein Netzwerk läuft, sollte es von Angreifern geschützt werden (Admin, sicheres Passwort, ...).
		\item Fehlertoleranz: Fehler sollten zuverlässig abgefangen und sinnvoll behandelt werden.
		\item Wiederherstellbarkeit: Bei Programmabsturz sollten zumindest die Daten des Spielers nicht verloren gehen (Gewinnpunkte etc.).
		\item Verständlichkeit: Das Spiel sollte leicht verständlich sein, da es auch von Kindern gespielt werden können soll.
		\item Erlernbarkeit: Personen, die das Spiel nicht kennen, sollen keine Probleme beim Erlernen der Spielregeln haben.
		Deswegen können die Regeln im Spiel und in der Lobby eingesehen werden. 
		\item Bedienbarkeit: Die Bedienbarkeit sollte leicht sein, da das Spiel von vielen Personengruppen gespielt werden soll. Dies wird durch eine intuitive GUI ermöglicht.
		\item Zeitverhalten: Es soll ein flüssiges Spielverhalten möglich sein.
		\item Verbrauchsverhalten: Der Ressourcenverbrauch soll durchschnittlich sein.
		\item Erweiterbarkeit: Erweiterungen sind nicht vorgesehen.
		\item Stabilität: Da keine Änderungen vorgesehen sind.
		\item Robustheit: Da das Spiel von verschiedensten Personen und insbesondere von Kindern gespielt werden soll, welche nicht mit den Spielregeln und der Bedienung des Programm vertraut sind, muss eine hohe Robustheit gewährleistet werden.
		\item Wartbarkeit: Nicht relevant für dieses Projekt.
		\item Plattformunabhängigkeit: Ist bereits durch Java gegeben.
\end{itemize}