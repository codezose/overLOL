\chapter{Systemtestfälle}

\newcounter{ptc}\setcounter{ptc}{10} %Pflichtenheft Test Counter
\newcounter{pstc}\setcounter{pstc}{1} %Pflichtenheft Substest Counter

\begin{description}[leftmargin=3em, style=sameline]
	%Variablen für Environment pftp:
	%  1. Test-Counter
	%  2. Substest-Counter
	%  3. Referenz auf Funktion, um die sich der Test dreht
	%  4. Referenz unter der dieser Test erreichbar sein soll
	%  5. Name des Tests
	\begin{pftp}{ptc}{pstc}{pproz:regist}{pproz:tregist}{Spieler registrieren}
		\item verfügbarer Nutzername  
		\item zu langer Benutzername
		\item zu kurzer Benutzername
		\item nicht verfügbarer Nutzername
		\item korrektes Passwort
		\item zu kurzes Passwort
		\item zu langes Passwort
	\end{pftp}
	
	\begin{pftp}{ptc}{pstc}{pproz:entfernen}{pproz:tentfernen}{Spieler entfernen}
		\item	Account löschen
		\item   mit gelöschte Daten anmelden
		\item	mit geloschte Daten neue Account registirieren
	\end{pftp}
		
	\begin{pftp}{ptc}{pstc}{pproz:anmeld}{ptest:tanmeld}{Spieler anmelden}
		\item registrierter Nutzername mit richtigem Passwort
		\item registrierter Nutzername mit falschem Passwort
		\item nicht registrierte Nutzername
	\end{pftp}
	
	\begin{pftp}{ptc}{pstc}{pproz:abmeld}{ptest:tabmeld}{Spieler abmelden}
		\item	während des Stichs/Spielpartie/Turnier
		\item	In der Lobby abmelden.
	\end{pftp}

	\begin{pftp}{ptc}{pstc}{pproz:erstel}{ptest:terstelt}{Spielraum erstellen}
		\item	ein Spielraum erstellen	
		\item	mehrere Spielräume (falsche)
		\item   Spielraum mit gleichem Namen erstellen
	\end{pftp}

	\begin{pftp}{ptc}{pstc}{pproz:schliessen}{ptest:tschliessen}{Spielraum schliessen}
		\item  Alle Spieler verlassen den Spielraum
	\end{pftp}

	\begin{pftp}{ptc}{pstc}{pproz:uebert}{ptest:tuebert}{Spielraumliste 	uebertragen}
		\item	Liste der bestehenden Spielräume anzeigen durch klicken	
	\end{pftp}
		
	\begin{pftp}{ptc}{pstc}{pproz:betret}{ptest:tbetret}{Spielraum beitreten}
		\item einem vollen Spielraum beitreten.
		\item in einen Spielraum beitreten in dem noch Platz ist
		\item Während eines Spiels beitreten 
		\item einem zweiten Spielraum betreten 
	\end{pftp}
		
	\begin{pftp}{ptc}{pstc}{pproz:verlas}{ptest:tverlas}{Spielraum verlassen}
		\item	vor,während und nach einem Spiel den Spielraum verlassen 
	\end{pftp}

	\begin{pftp}{ptc}{pstc}{pproz:anzeig}{ptest:tanzeig}{Spielstand anzeigen}
		\item	nach einer Spielpartie
	\end{pftp}
		
	\begin{pftp}{ptc}{pstc}{pproz:anzeig2}{ptest:tanzeig2}{Spielstatistik anzeigen}
		\item	Statistik in der Lobby aufrufen
	\end{pftp}
		
	\begin{pftp}{ptc}{pstc}{pproz:statistikberechnen}{ptest:tstatistikberechnen}{Highscore berechnen}
		\item	Spielpartie spielen und den Highscore danach pruefen
	\end{pftp}
		
	\begin{pftp}{ptc}{pstc}{pproz:anzeig3}{ptest:tanzeig3}{Punktestand anzeigen}
		\item	Punktestandanzeige pruefen
	\end{pftp}
		
	\begin{pftp}{ptc}{pstc}{pproz:punkteberechnen}{ptest:tpunkteberechnen}{Punktestand berechnen}
		\item	Punktestand pruefen auf korrektheit(Fuer alle Karten in allen Spielmodi)
	\end{pftp}

	\begin{pftp}{ptc}{pstc}{pproz:gewinnpunkteberechnen}{ptest:tgewinnpunkteberechnen}{Gewinnpunkte berechnen}
		\item	Alle nach Spielregeln moeglichen Faelle
	\end{pftp}

	\begin{pftp}{ptc}{pstc}{pproz:hinzuf}{ptest:thinzuf}{Bot hinzufuegen}
		\item	Bot bei 3 Spielern hinzufuegen
		\item	Bot bei weniger als 3 Spielern hinzufuegen
	\end{pftp}
		
	\begin{pftp}{ptc}{pstc}{pproz:entfer}{ptest:tentfer}{Bot entfernen}
		\item	Bot entfernen
		\item	Bei 0 Bots den Bot entfernen
	\end{pftp}
		
	\begin{pftp}{ptc}{pstc}{pproz:notfallbot}{ptest:tnotfallbot}{Notfallbot springt ein}
		\item	Timer ablaufen lassen
		\item	Notfallbot (mehrere Spielpartien) weiterspielen lassen
		\item   Notfallbot beendet Spiel
	\end{pftp}
		
	\begin{pftp}{ptc}{pstc}{pproz:chatte1}{ptest:tchatte1}{Chatnachricht im Spiel senden}
		\item	Nachrichten verschiedener Länge versenden
		\item	Nachricht empfangen
	\end{pftp}

	\begin{pftp}{ptc}{pstc}{pproz:chatte2}{ptest:tchatte2}{Chatnachricht in der Lobby senden}
		\item analog zu oben
	\end{pftp}
		
	\begin{pftp}{ptc}{pstc}{pproz:starte}{ptest:tstarte}{Spiel starten}
		\item	mit 3 Mitspielern
		\item	mit weniger als 3 Mitspielern
		\item	mit Bots und Spielern
	\end{pftp}
		
	\begin{pftp}{ptc}{pstc}{pproz:start}{ptest:tstart}{Neue Spielpartie starten}
		\item	Spielpartie im Spiel starten
		\item	Spielpartie wird von nicht Admin gestartet
	\end{pftp}
		
	\begin{pftp}{ptc}{pstc}{pproz:mischen}{ptest:tmischen}{Karten mischen}
		\item	Karten hintereinander mehrmals mischen
	\end{pftp}
		
	\begin{pftp}{ptc}{pstc}{pproz:sortieren}{ptest:tsortieren}{Handkarten sortieren}
		\item	Handkarten sortieren lassen
	\end{pftp}
		
	\begin{pftp}{ptc}{pstc}{pproz:austei}{ptest:taustei}{Karten austeilen}
		\item	Karten austeilen lassen
	\end{pftp}
		
	\begin{pftp}{ptc}{pstc}{pproz:ausspi}{ptest:tausspi}{Karte ausspielen}
		\item eine entsprechend der Spielregeln passende Karte ausspielen.
		\item eine Karte, die laut Spielregeln nicht gespielt werden darf, ausspielen.
		\item eine Trumpfkarte ausspielen.
	\end{pftp}
		
	
		
	\begin{pftp}{ptc}{pstc}{pproz:trumpf}{ptest:ttrumpf}{Trumpfkarte zeigen}
		\item	Trumpfkarte wird angezeigt.
	\end{pftp}
		
	\begin{pftp}{ptc}{pstc}{pproz:drueck}{ptest:tdrueck}{Karten drücken}
		\item	jede mögliche Karte drücken
	\end{pftp}	
		
			
	\begin{pftp}{ptc}{pstc}{pproz:stichb}{ptest:tstichb}{Stich beenden}
		\item	Stich beenden und überprüfen, ob Stichkarten richtig zugeteilt werden
	\end{pftp}	
		
	\begin{pftp}{ptc}{pstc}{pproz:partieb}{ptest:tpartieb}{Spielpartie beenden}
		\item Beenden, wenn Spieler weniger als 7 Gewinnpunkte haben
		\item Beenden, wenn Spieler mindestens 7 Gewinnpunkte hat
	\end{pftp}	
		
	\begin{pftp}{ptc}{pstc}{pproz:spielb}{ptest:tspielb}{Spiel beenden}
		\item Beenden, wenn Spieler weniger als 7 Gewinnpunkte haben
		\item Beenden, wenn Spieler mindestens 7 Gewinnpunkte hat
	\end{pftp}	
		
	\begin{pftp}{ptc}{pstc}{pproz:regelnanzeigen}{ptest:tregelnanzeigen}{Spielregeln anzeigen}
		\item	Überprüfen, ob Spielregeln korrekt angezeigt werden
	\end{pftp}	
\end{description}		
